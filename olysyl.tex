\documentclass[a4paper,10pt]{paper}

\usepackage[utf8]{inputenc}
\usepackage[italian]{babel}
\usepackage[T1]{fontenc}
\usepackage{lmodern}
\usepackage[utf8]{inputenc}
\usepackage{graphicx}

\usepackage{hyperref}

\date{}
\title{Programma non ufficiale per le Olimpiadi Italiane di Matematica}
\subtitle{Versione 0.3}
\author{Federico Poloni}

\begin{document}
 \maketitle
 
 Questo documento contiene una lista non ufficiale degli argomenti che è utile sapere per risolvere i problemi di queste tre fasi delle Olimpiadi Italiane di Matematica:
 \begin{itemize}
  \item i giochi di Archimede;
  \item la gara provinciale di Febbraio;
  \item la finale nazionale di Cesenatico.
 \end{itemize}

 Non è un ``programma'' ufficiale. È ricavato dalla mia esperienza come concorrente e organizzatore. Questo è quello che ci aspettiamo che un concorrente preparato veda in qualche momento della sua carriera scolastica.
 
 Stilare un programma ufficiale sarebbe impossibile. Qualche volta può capitare che in un problema serva un caso particolare di un teorema molto più complicato, e in realtà quello che ci si aspetta è che un concorrente sveglio non si ``porti da casa'' il teorema, ma arrivi sul momento a capire quello che succede in quel caso particolare più semplice di quello generale. Qualche volta un teorema più avanzato compare in una soluzione alternativa, che può sembrare più naturale a chi già padroneggia il teorema; questo non vuol dire che il teorema sia necessario.

 Maggiori dettagli sul tema ``cosa serve sapere nelle gare'' e ``come prepararsi'' si trovano in fondo a questo documento.
 
 
\section{Postfazione}
 
\paragraph{Conoscenze e idee} Ci sono due situazioni in cui si può trovare uno studente che partecipa alle Olimpiadi di Matematica. Alcuni, non avendo mai sentito parlare di Olimpiadi, vedono qualche problema, notano che è diverso da quelli dei libri di testo di matematica in cui basta leggere il capitolo prima per capire al volo come si fanno, e si chiedono: ``Cosa devo studiare per farli? Ho già visto tutto quello che serve? Non l'abbiamo fatto a scuola!''. Altri magari sono capitati sul forum delle Olimpiadi~\cite{oliforum} o a uno stage locale, o hanno visto le schede Olimpiche di Massimo Gobbino~\cite{schedeolimpiche}, hanno visto persone più esperte usare una quantità mastodontica di teoria, e si chiedono: ``la gente qui parla di disuguaglianza di Cauchy-Schwarz e di Nullstellensatz combinatorico come se fossero cose semplicissime, e io che non le so come mai farò  a Cesenatico?''.
 
 In entrambi i casi la risposta è la stessa: per risolvere i problemi di Cesenatico servono molte meno conoscenze di quanto può sembrare. Per esempio i due argomenti citati sopra non servono proprio a nulla. O meglio: niente è \emph{completamente} inutile, e sapere qualcosa di più può magari portare a una soluzione alternativa, ma chi prepara le gare cerca sempre di fare in modo che ci sia una soluzione ufficiale alla portata del concorrente. Spesso, non ci si aspetta che \emph{sappiate già} un teorema che smonta l'esercizio in una riga, ma che \emph{capiate} cosa succede in quel caso particolare durante la gara.
 
 Quindi quello che serve, più che le conoscenze, sono le \emph{idee}, perché qui ogni esercizio si fa in un modo diverso. Sono più difficili da insegnare dei teoremi, e il modo migliore di impararle è vedendo tanti problemi, facendo esperienza e cercando di generalizzare i metodi usati. 
 

\paragraph{La via alta e la via bassa}
Le OdM in Italia in questo momento sono strutturate su due percorsi paralleli:
\begin{description}
\item[La via bassa] I giochi di Archimede, la gara provinciale di Febbraio, la finale nazionale di Cesenatico. Gli organizzatori cercano il più possibile di fare in modo che gli esercizi di questa parte siano risolvibili a persone che conoscono il solo programma scolastico o poco più, concentrandosi più sulle idee. Può succedere che un argomento avanzato (come il Cauchy-Schwarz di sopra) possa portarvi a una soluzione alternativa, o farvi venire in mente le idee giuste, ma questo spesso è non voluto. Durante la selezione dei problemi, vengono regolarmente \emph{scartati} esercizi con motivazioni come ``Questo non possiamo darlo, serve sapere X'', oppure ``Sapere X aiuta molto nella soluzione, chi lo sa è troppo avvantaggiato'', o ``Questo si fa in una riga con questo risultato avanzato, meglio non metterlo''. Insomma, cerchiamo di mettere tutti sullo stesso livello e non richiedere troppo studio. Se guardate i testi delle gare nazionali di altri paesi, raccolti per esempio su~\cite{mathlinks}, noterete che spesso contengono esercizi molto più tecnici (disuguaglianze, per esempio), e l'Italia è un'eccezione.
 
\item[La via alta] Lo stage senior e gli altri stage di preparazione che si svolgono a Pisa (ma \emph{non} quelli a livello provinciale che si svolgono in vari distretti in Italia), le gare internazionali e le Olimpiadi di Matematica. Questo è un percorso scollegato dal resto, dedicato a chi vuole approfondire (e di molto) le tecniche di problem-solving e ambire a rappresentare l'Italia in una delle gare internazionali a cui partecipiamo.

Come accennavamo sopra, gli altri paesi solitamente non ci ricambiano la cortesia, quindi bisogna imbracciare l'artiglieria. Per affrontare i problemi delle gare internazionali serve un bagaglio di tecniche e conoscenze ben più ampio: indicativamente, tutto quello che trovate nelle Schede Olimpiche di Massimo Gobbino~\cite{schedeolimpiche}, e anche di più, soprattutto in termini di idee. Anche per chi è brillante nelle gare della ``via bassa'' è necessario un momento di studio per imparare questi concetti. Cerchiamo di insegnare quello che serve nello \emph{stage Senior}, che si svolge a Settembre a Pisa. Vengono invitati a spese dell'UMI gli studenti che hanno ottenuto risultati eccellenti a Cesenatico, con un occhio particolare verso i giovani promettenti, ma è possibile partecipare anche da volontari, gratuitamente, risolvendo (a casa) alcuni problemi di ammissione. L'annuncio di questi stage viene fatto sul forum~\cite{oliforum}, nella sezione Olimpiadi Della Matematica.
\end{description}

\paragraph{Troppe conoscenze possono essere dannose}
Come detto sopra, il maggiore bagaglio di conoscenze è utile solo a chi vuole fare una carriera da problem-solver di livello internazionale. Imparare teoremi dal nome altisonante non serve a molto, e anzi dopo il primo stage Senior molti hanno un piccolo calo di prestazioni nelle gare della ``via bassa'', perché perdono tempo provando ad applicare con insistenza gli strumenti nuovi, invece di cercare le idee ingegnose che servono per quei problemi. Come dice un detto inglese, \emph{quando hai in mano un martello, tutto sembra assomigliare a un chiodo}.

\paragraph{Quindi, come mi preparo per Cesenatico?}
Studiando quella poca teoria che serve sui manuali scolastici, o su libri come \cite{culipiccoli,dispenseudine,paolini}, provando a fare tanti esercizi, leggendo le soluzioni di quelli che non ci vengono (solo dopo aver provato a risolverli per un'ora o più!), chiedendo aiuto su internet~\cite{oliforum}, a un collega più grande, o a un insegnante/allenatore.
 
 \paragraph{Il percorso che porta all'illuminazione}
 \begin{description}
  \item[Prima fase] Leggere problemi di livello pari o poco superiore a quello per cui ci si prepara; provare a farli come indicato nel paragrafo qui sopra; quando si fallisce, leggere le soluzioni ufficiali e dire ``Cosa? Non mi sarebbe mai venuto in mente!''.
  \item[Seconda fase] Dopo aver visto un po' di esercizi, riconoscere che ``Ehi, questo tutto sommato è simile a un'idea che ho già visto in un vecchio esercizio; tutto sommato poteva venirmi in mente, perché non è successo?''. Risolvere ancora esercizi simili, ma con domandandosi ``Cosa mi ricorda questo esercizio? Cosa posso provare a fare qui?''.
  \item[Terza fase] Riconoscere che questo succede sempre più spesso. Negli esercizi ricompaiono tutto sommato idee simili. Riconoscere che facendo passare tutte le idee già viste una per una e provando ad applicarle, avendo a disposizione una settimana e provando tutto quello che devo provare, quell'idea ci sarebbe venuta.
  \item[Quarta fase] \dots{}e ora, è solo questione di tempo, no? Affinare le abilità, capire come farsi venire più in fretta quelle idee, e fare gli esercizi in tempo di gara.
 \end{description} 
 
\end{document}
