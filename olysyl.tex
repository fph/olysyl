\documentclass[a4paper,10pt]{paper}

\usepackage[utf8]{inputenc}
\usepackage[italian]{babel}
\usepackage{csquotes}
\usepackage[T1]{fontenc}
\usepackage{lmodern}
\usepackage[utf8]{inputenc}
\usepackage{graphicx}
\usepackage{epigraph}

\usepackage{amsmath,amsthm,amsfonts,mathtools}
\usepackage[backend=biber,style=alphabetic]{biblatex}
\bibliography{olysyl}

\renewcommand{\star}{(*)}
\newcommand{\sstar}{(**)}

\usepackage{microtype}
\usepackage{hyperref}

\date{}
\title{Programma non ufficiale per le Olimpiadi Italiane di Matematica}
\subtitle{Versione 0.4 -- Ottobre 2014}
\author{Federico Poloni \href{mailto:fpoloni@di.unipi.it}{\texttt{<fpoloni@di.unipi.it>}}}

\begin{document}
 \maketitle
 
 Questo documento contiene una lista non ufficiale degli argomenti che è utile sapere per risolvere i problemi di queste tre fasi delle Olimpiadi Italiane di Matematica:
 \begin{itemize}
  \item i giochi di Archimede;
  \item la gara provinciale di Febbraio;
  \item la finale nazionale di Cesenatico.
 \end{itemize}

 Non è un documento ufficiale. È ricavato dalla mia esperienza come concorrente e organizzatore. Questo è quello che ci si aspetta che un concorrente preparato veda in qualche momento della sua carriera scolastica.
 
 Stilare un programma ufficiale sarebbe impossibile. Qualche volta può capitare che serva un caso particolare di un teorema più complicato, e in realtà quello che ci si aspetta è che un concorrente sveglio non si ``porti da casa'' il teorema, ma arrivi sul momento a capire quello che succede in quel caso particolare. Qualche volta un teorema più avanzato compare in una soluzione alternativa, che può sembrare più naturale a chi già padroneggia il teorema; questo non vuol dire che il teorema sia necessario.

 Maggiori dettagli sul tema ``cosa serve sapere nelle gare'' e ``come prepararsi'' si trovano in fondo a questo documento. TL;DR: \textbf{non studiate troppa teoria; fate molti problemi e leggete molte soluzioni}.
 
\section*{Legenda}
Sono segnati con una stella \star{} alcuni argomenti extra che capitano solo in pochi problemi. Sono indicati con due stelle \sstar{} argomenti ``ancora più extra'' che solitamente \emph{non} sono considerati fatti noti ai concorrenti (anzi, alcuni problemi vengono scartati perché ``per farlo bisogna sapere per forza X''), ma che possono capitare in qualche raro caso, per esempio nei problemi di Cesenatico con numeri più alti. Dubito che servano prima di Cesenatico.

Gli argomenti senza stelline o con una stellina sola dovrebbero essere tutte (o quasi) cose insegnate nei primi anni delle scuole superiori; un'eccezione sono le congruenze, che pur capitando di frequente nelle gare non vengono insegnate in molte scuole.

\section{Algebra}
\begin{itemize}
 \item Saper applicare le relazioni algebriche date per ipotesi nei casi più utili --- per esempio, se mi viene data un'equazione con $f(x,y)$, calcolare $f(0,0)$ o $f(1,1)$ o $f(x,x)$.
 \item Avere una discreta manualità nello svolgere i conti senza perdersi, per esempio esplicitare un'equazione in funzione di una delle variabili che compaiono.
 \item $0$ non è un numero positivo!
 \item Formula per la somma dei primi $k$ interi. Formule per la somma di progressioni aritmetiche e geometriche.
 \item Proprietà delle potenze (es. $a^n\cdot a^m = a^{n+m}$).
 \item Fattorizzare polinomi e raccogliere espressioni comuni.
 \item Scomposizioni notevoli: $(a^n \pm b^n) / (a\pm b)$.
 \item Potenze di un binomio e triangolo di Tartaglia (anche detto di Pascal).
 \item Quadrato di un trinomio.
 \item Grado di un polinomio e suo comportamento rispetto alle operazioni.
 \item Manipolazione di polinomi. Divisione tra polinomi (con resto). Regola di Ruffini (se $p(a)=0$, allora $p(x)$ è multiplo di $x-a$).
 \item Relazioni tra radici e coefficienti di un polinomio di secondo, massimo terzo grado (anche dette formule di Viète).
 \item \sstar{} Principio di identità dei polinomi (se due polinomi di grado $\leq n$ assumono lo stesso valore quando vengono valutati in $n+1$ punti, allora hanno gli stessi coefficienti).
 \item Saper impostare e risolvere sistemi di due, tre (dis)equazioni di primo grado
 \item Saper manipolare in modo furbo sistemi di equazioni di primo grado (per esempio, sommando e sottraendo un'equazione all'altra).
 \item Equazioni di secondo grado: formula risolutiva.
 \item \star Sistemi simmetrici di due equazioni in due incognite: $x+y=a$, $xy=b$.
 \item Valore assoluto: definizione, comportamento, alcuni utilizzi. Disuguaglianza triangolare (nel senso algebrico del termine).
 \item Disuguaglianze: comportamento di una disuguaglianza rispetto ad alcune operazioni (sommare, moltiplicare uno stesso termine). Impostare semplici catene di disuguaglianze.
 \item Disuguaglianze: un quadrato è maggiore di zero. \star{} Segno di un polinomio di secondo grado e discriminante.
 \item \sstar{} Derivate di polinomi; calcolo ``a cannonate'' di massimi e minimi con le derivate. \emph{Solitamente mai necessari; c'è sempre un modo algebrico alternativo di fare le stesse cose.}
\end{itemize}

\section{Teoria dei numeri}
\begin{itemize}
 \item Tutta l'algebra della sezione precedente\dots
 \item Rappresentazione in base: se un numero si scrive in base 10 come $abc$, allora è $100a+10b+c$.
 \item Conti con le frazioni; passare da numero con la virgola a frazione e viceversa.
 \item Definizione di numero primo. $1$ non è primo!
 \item Divisione con resto.
 \item Criteri di divisibilità: $2,3,4,5,9,10,11$.
 \item Fattorizzazione di numeri interi (a occhio, per tentativi o con i criteri di divisibilità).
 \item Proprietà della divisibilità: se $d$ divide $a$ e $b$, allora divide anche $ka$, $a+b$, $a-b$. Se $d$ divide tutti i termini di un'equazione del tipo $x+y-z=t$ a parte al più uno, allora divide anche l'ultimo.
 \item Massimo comun divisore e minimo comune multiplo. Definizione e calcolo.
 \item Operazioni con numeri pari e dispari (es. $P\cdot D=P$).
 \item Comportamento dell'ultima cifra rispetto alle operazioni (cioè, congruenze modulo 10).
 \item Idea intuitiva di ``resto della divisione'' e suo comportamento.
 \item \star Congruenze in generale, comportamento rispetto a somme e prodotti.
\end{itemize}

\section{Geometria}
\begin{itemize}
 \item ``Angle-chasing'' e ``segment-chasing'': segnare angoli e segmenti uguali su una figura, calcolare l'ampiezza di alcuni angoli in funzione di altri.
 \item Fatti utili per l'angle-chasing: somma degli angoli interni di un triangolo, relazioni tra angoli e segmenti determinati da due rette parallele (teorema di Talete), angoli adiacenti e opposti al vertice, un triangolo ha due angoli uguali se e solo se ha due lati uguali.
 \item Conti con i segmenti: se questo misura $a$ e quest'altro $b$, allora questo vale\dots
 \item Criteri di uguaglianza e similitudine per triangoli.
 \item Proprietà di triangoli isosceli ed equilateri
 \item \star{} Centri di un triangolo: ortocentro, baricentro, circocentro, incentro.
 \item \star{} Formule per il raggio del cerchio inscritto e circoscritto in un triangolo.
 \item Teorema della bisettrice (la bisettrice di un angolo di un triangolo taglia il lato opposto in segmenti proporzionali ai lati adiacenti).
 \item Disuguaglianza triangolare.
 \item Teoremi di Pitagora ed Euclide.
 \item 3,4,5 è una terna pitagorica.
 \item Classificazione dei quadrilateri (trapezi, parallelogrammi\dots). Loro proprietà (ad esempio: in un quadrilatero, le diagonali si bisecano se e solo se è un parallelogramma)
 \item \star{} Criteri di inscrittibilità e circoscrittibilità dei quadrilateri.
 \item Teoremi delle due secanti, delle due corde, della secante e della tangente (anche noti come: potenza di un punto rispetto a una circonferenza).
 \item Relazioni tra angoli al centro e angoli alla circonferenza.
 \item Oggetti perpendicolari in una circonferenza: raggio e tangente, corda e congiungente al centro per il suo punto medio.
 \item \sstar{} Circonferenza di Apollonio (fissati $k\neq 1$ e due punti $A$, $B$, il luogo dei punti $P$ tali che $\frac{AP}{BP}=k$ è una circonferenza).
 \item Comportamento dell'area rispetto al ``tagliare'', spostare e ``reincollare'' pezzi di figure piane.
 \item Aree e perimetri notevoli: triangolo, trapezio, parallelogramma, rettangolo, cerchio, \star{} esagono regolare.
 \item Triangoli notevoli: $30^\circ-60^\circ-90^\circ$, $90^\circ-45^\circ-45^\circ$.
 \item \sstar{} Formula di Erone.
 \item Rotazioni e simmetrie assiali.
 \item \star{} Similitudini (omotetie). Comportamento di lunghezze e aree sotto similitudine.
 \item \star{} Definizione di luogo; aver visto almeno una volta nella vita una dimostrazione (sintetica!) della caratterizzazione di un luogo. Necessità di dimostrare entrambi i versi (se $P$ sta nel luogo, allora soddisfa la proprietà; se $P$ non sta sul luogo, allora non la soddisfa).
 \item Caratterizzazione di assi e bisettrici come luoghi.
 \item Geometria solida: nozione e visione intuitiva in 3D dei solidi.
 \item  Volume del prisma e della piramide (inclusi cono e cilindro).
 \item Idea di piano che taglia una figura solida: trovare proprietà della geometria piana (ad esempio similitudini) all'interno di pezzi di figure solide.
 \item Aver visto almeno una volta nella vita un problema che usa l'idea di fare uno sviluppo di un solido sul piano, e saper applicare quell'idea.
 \item \star{} Geometria analitica: idea intuitiva di ``insieme dei punti che soddisfano una proprietà sul piano cartesiano''. Equazione della retta per due punti; equazione della circonferenza dati centro e raggio; incrociare rette e circonferenze. \emph{Solitamente solo per Archimede triennio}.
 \item \star{} Accorgersi di quando si può applicare la geometria analitica con profitto e quando conduce solo a un mare di conti. Capire come vengono i conti prima di farli davvero.
\end{itemize}

\section{Combinatoria}
\begin{itemize}
 \item Concetto intuitivo di ``quanti modi''; significato della somma e del prodotto (due scelte, una in 3 modi e una in 5 danno 15 modi. Due possibilità alternative, una da 5 modi e una da 3, danno 8 modi).
 \item Conteggi notevoli: in quanti modi posso scegliere un sottoinsieme di $k$ elementi su $n$, oppure una $k$-upla ordinata (anche noti in campo scolastico come ``disposizioni'', ``combinazioni'', ``permutazioni'').
 \item Avere visto esempi conteggi più complessi ridotti a casi più semplici: anagrammi con ripetizioni, mani di poker.
 \item Definizione di fattoriale e di coefficiente binomiale. \star{} Alcune proprietà dei coefficienti binomiali.
 \item Insiemistica: disegnare diagrammi di Venn e contare gli elementi in ogni regione.
 \item \star{} Avere visto almeno una volta nella vita una dimostrazione che faccia uso di una colorazione.
 \item \sstar{} Idea di \emph{double counting} (averlo visto almeno una volta nella vita).
 \item Probabilità: definizione come casi favorevoli/casi possibili; calcolo come due ``prendo tanti casi equiprobabili e applico la definizione''.
\end{itemize}

\section{Altro}
\begin{itemize}
 \item Logica elementare: e (AND), o (OR), se\dots{}allora, condizioni necessarie o sufficienti. Dimostrare $A \implies B$ attraverso $\operatorname{not}(B)\implies \operatorname{not}(A)$. Aver visto una volta nella vita una dimostrazione per assurdo.
 \item Nella dimostrazione di un ``se e solo se'' bisogna far vedere entrambe le direzioni.
 \item Similmente, per dimostrare che due insiemi sono uguali (``le soluzioni dell'equazione X sono Y'') servono due ``direzioni'' (ogni soluzione di X è un Y, e ogni Y risolve X).
 \item Aver visto un problema di cavalieri e furfanti.
 \item Saper riempire una ``tavola di verità''.
 \item \star{} Idea intuitiva di dimostrazione per induzione (``e così via'').
 \item \sstar{} Induzione formalizzata un po' meglio.
 \item Idea intuitiva di principio dei cassetti (anche noto come \emph{pigeonhole principle} o principio di Dirichlet): se ho $n+1$ palline divise tra $n$ scatole\dots
 \item Formule di fisica elementare per il moto a velocità uniforme: $s=s_0+vt$. Definizione di velocità e accelerazione.
 \item Disegnare diagrammi spazio/tempo e interpretarli (velocità $=$ pendenza della retta\dots).
\end{itemize}

\section*{Postfazione}
 
\paragraph{Conoscenze e idee} Ci sono due situazioni opposte in cui si può trovare uno studente che affronta per la prima volta le Olimpiadi di Matematica. Alcuni vedono i problemi, notano che sono diverso da quelli dei libri di testo di matematica in cui basta leggere il capitolo prima per capire come si fanno a colpo sicuro, e si chiedono: \emph{``Cosa devo studiare? Non ho mai visto questi problemi; non li abbiamo fatti a scuola!''}. Altri magari sono capitati sul forum delle Olimpiadi~\cite{oliforum} o a uno stage locale, o hanno visto un libro come~\cite{schedeolimpiche}, o hanno visto altri concorrenti usare una quantità mastodontica di teoria, e si chiedono: \emph{``La gente qui parla di disuguaglianza di Cauchy-Schwarz e di Nullstellensatz combinatorico come se fossero cose semplicissime, e io che non le so come mai farò?''}.
 
 La risposta è la stessa: per risolvere i problemi di Cesenatico servono molte meno conoscenze di quanto può sembrare. Per esempio i due teoremi appena citati non servono a nulla. O meglio: niente è \emph{completamente} inutile; magari portano a una soluzione alternativa di un problema, ma chi prepara le gare cerca sempre di fare in modo che ci sia una soluzione alla vostra portata. Non ci si aspetta che \emph{sappiate già} un metodo o un teorema che smonta l'esercizio in una riga (se davvero c'è), ma che \emph{capiate} cosa succede in quel caso particolare e \emph{inventiate} una soluzione nuova durante la gara. L'accento non è sull'applicare le stesse tecniche ad esercizi tutti uguali, ma sull'aver visto alcune idee e saperle usare in contesti nuovi.
 
 Ogni esercizio si fa in un modo diverso; quello che serve, più che le conoscenze, sono le \emph{idee}. Sono più difficili da insegnare di teoremi e metodi preconfezionati; il modo migliore di impararle è vedendo tanti problemi, facendo esperienza e cercando di generalizzare i metodi usati. 

\paragraph{Quindi, come mi preparo per Cesenatico?}
Studiando quella poca teoria che serve sui testi scolastici o su libri come \cite{culipiccoli,dispenseudine,paolini}, provando a fare tanti esercizi, leggendo le soluzioni di quelli che non ci vengono (solo dopo aver provato a risolverli per un'ora o più!), chiedendo aiuto su internet~\cite{oliforum}, a un collega più grande, o a un insegnante/allenatore.
 
 \paragraph{Il percorso che porta all'illuminazione}
 \begin{description}
  \item[Prima fase] Leggere problemi di livello pari o poco superiore a quello per cui ci si prepara; provare a farli come indicato nel paragrafo qui sopra; quando si fallisce, leggere le soluzioni ufficiali e dire ``Cosa? Non mi sarebbe mai venuto in mente!''.
  \item[Seconda fase] Dopo aver visto un po' di esercizi, riconoscere che ``Ehi, questo tutto sommato è simile a un'idea che ho già visto in un vecchio esercizio; tutto sommato poteva venirmi in mente, perché non è successo?''. Risolvere ancora esercizi simili, ma con domandandosi ``Cosa mi ricorda questo esercizio? Cosa posso provare a fare qui?''.
  \item[Terza fase] Riconoscere che questo succede sempre più spesso. Negli esercizi ricompaiono tutto sommato idee simili. Riconoscere che facendo passare tutte le idee già viste una per una e provando ad applicarle, avendo a disposizione una settimana e provando tutto quello che devo provare, quell'idea ci sarebbe venuta.
  \item[Quarta fase] \dots{}e ora, è solo questione di tempo, no? Affinare le abilità, capire come farsi venire più in fretta quelle idee, e fare gli esercizi in tempo di gara.
 \end{description} 

\paragraph{Delusioni} Le Olimpiadi di Matematica sono come una gara sportiva; si compete cercando di ottenere i risultati migliori, e i vari livelli sono via via più difficili. È normale arrivare a un certo livello, diverso da persona a persona, e poi, nonostante tutto l'impegno e il tempo dedicato, trovare difficoltà. Questo sarà accompagnato inevitabilmente da un attimo di delusione. Indipendentemente dall'aspetto competitivo, speriamo che apprezzerete di aver incontrato tramite queste gare una matematica più creativa e interessante del solito, e conosciuto molte persone che condividono la vostra passione. Per usare una frase fatta, \emph{l'importante è partecipare}.

\paragraph{Olimpiadi e matematica vera}
 La matematica che si incontra all'università e nella ricerca ha qualche aspetto simile a quella delle Olimpiadi, ma è una materia di più ampio respiro: i tempi sono molto più lunghi e ha un'importanza maggiore saper interiorizzare concetti astratti e porsi le domande giuste. Le Olimpiadi invece consistono ``solo'' nel saper risolvere velocemente problemi tutto sommato abbordabili: per fare un paragone atletico, è come cercare di paragonare la maratona e i cento metri di corsa. Un maratoneta ha una buona condizione fisica e tutto sommato se la caverà benino nei cento metri, ma non per forza al livello dei primi 300 in Italia che si allenano apposta per quella gara; viceversa, un campione dei 100 metri correrà una maratona discreta, ma non per forza da record del mondo.
 
 Quindi, non traete indicazioni negative sul vostro talento in matematica dal fatto di non riuscire ad arrivare tra i primi in Italia in queste gare; l'insieme $\{\text{matematici di successo}\}$ e quello $\{\text{ex-concorrenti delle olimpiadi}\}$ non coincidono; moltissimi matematici di professione non hanno mai partecipato a queste gare o non hanno mai ottenuto grandi successi.
 
 Traete, semmai, indicazioni \emph{positive}: se vi interessate alle olimpiadi, vuol dire che fare matematica tutto sommato non vi dispiace; perché non continuare?

\paragraph{La via alta e la via bassa}
Le OdM in Italia in questo momento sono strutturate su due percorsi paralleli:
\begin{description}
\item[La via bassa] I giochi di Archimede, la gara provinciale di Febbraio, la finale nazionale di Cesenatico. Queste sono le manifestazioni a cui dovete puntare come nuovi concorrenti. Gli organizzatori cercano il più possibile di fare in modo che gli esercizi di questa parte siano affrontabili da persone che conoscono il solo programma scolastico o poco più, concentrandosi più sul trovare le idee risolutive. Quando scegliamo i problemi, \emph{scartiamo} regolarmente esercizi con motivazioni come \emph{``Sapere X aiuta molto nella soluzione, chi lo sa è troppo avvantaggiato''}, o \emph{``Questo si fa in una riga con questa tecnica, meglio non metterlo''}. Insomma, cerchiamo di mettere tutti sullo stesso livello e non richiedere troppo studio. Se guardate i testi delle gare nazionali di altri paesi, raccolti per esempio su~\cite{mathlinks}, noterete che spesso contengono esercizi molto più tecnici (disuguaglianze, per esempio), e l'Italia è un'eccezione.
 
\item[La via alta] Gli stage di preparazione che si svolgono a Pisa (Stage Senior, Winter Camp, PreIMO; \emph{non} quelli a livello provinciale che si svolgono in varie città in Italia), le gare internazionali e le IMO (International Mathematical Olympiad). Questo è un livello più avanzato, dedicato a chi vuole approfondire, e molto, le tecniche di problem-solving e ambire a rappresentare l'Italia in una delle gare internazionali a cui partecipiamo. È quasi completamente separato dalle gare precedenti, se non per il fatto che i concorrenti che ottengono buoni risultati a Cesenatico vengono invitati a questi stage a spese dell'Unione Matematica Italiana.

Per competere in gare di questo tipo in un contesto internazionale, il bagaglio di conoscenze e tecniche di problem-solving che serve è maggiore, quindi bisogna imbracciare l'artiglieria. Anche per chi è brillante nelle gare della ``via bassa'' è necessaria una buona quantità di studio. Cerchiamo di insegnare quello che serve nello \emph{stage Senior}, che si svolge a Settembre a Pisa. Vengono invitati a spese dell'UMI gli studenti che hanno ottenuto risultati eccellenti a Cesenatico, con un occhio particolare verso i giovani promettenti, ma è possibile partecipare anche da volontari, gratuitamente. L'annuncio di questi stage viene fatto sul nostro forum~\cite{oliforum}, nella sezione Olimpiadi Della Matematica. Un testo che raccoglie in modo stringato un po' della teoria che serve a questo livello sono le Schede Olimpiche di Massimo Gobbino~\cite{schedeolimpiche}. 
\end{description}

\paragraph{Troppe conoscenze possono essere dannose}
Questo maggiore bagaglio di conoscenze è utile solo a chi vuole fare una carriera da problem-solver di livello internazionale, e quindi spesso già se la cava bene nelle gare nazionali. Mandare a memoria teoremi dal nome altisonante, da solo, non serve a molto; anzi dopo il primo stage ``alto'' molti hanno un calo di prestazioni anche nelle gare normali, perché perdono tempo provando ad applicare con insistenza gli strumenti nuovi, invece di cercare le idee ingegnose che servono per questi problemi. Come dice un detto inglese, \emph{quando hai in mano un martello, tutto assomiglia a un chiodo}.
 
\printbibliography
 
\end{document}
